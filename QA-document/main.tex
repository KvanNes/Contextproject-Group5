\documentclass{article}
\newcounter{question}
\setcounter{question}{0}
\usepackage[utf8]{inputenc}

\title{QA}
\author{Gaming project group 5}
\date{}

\begin{document}

\newcommand\Que[1]{%
   \leavevmode\par
   \stepcounter{question}
   \noindent
   \thequestion. Q --- #1\par}

\newcommand\Ans[2][]{%
    \leavevmode\par\noindent
   {\leftskip37pt
    A --- \textbf{#1}#2\par}}


\maketitle

\section{Introduction}
This testing session is meant to test the racing game DuoDrive\texttrademark. We will ask you to read the instructions of the game and play the game. After that we would be pleased if you would answer our questions being asked in this document.

\section{Instructions}
\subsection{Goal}
The goal of the game is to reach the finish before the other team(s). 
\subsection{How}
The way you will reach the finish is in a car that you will drive with 2 persons. One of the persons will be in control of throttle, and the other will be in control of the steering. 
\subsection{communication}
The only way to reach the finish before the teams is with excellent communication. This is because both players have limited accessibility to the game. 
\subsubsection{Steering player}
The steering player has limited view of the road, so he does not know where he must go. This information he needs to get from the throttle controlling player, by means of shouting.
\subsubsection{Throttle controlling player}
This player controls the throttle. He has no direct vision of the car, instead he sees an overview of the track with a small dot as track. He has some indication how fast the car must go, but not precise. This information he must get from the steering player.
\subsection{Controls PC}
\subsubsection{steering player}
use the right and left arrow key to steer the car
\subsubsection{Throttle controlling player}
use the up and down arrow key to control the speed of the car.
\subsection{Controls android}
\subsubsection{Steering player}
Use the gyroscope in your android device to steer the car, thus turn your device to left and right to steer in the direction you want.
\subsubsection{Throttle controlling player}
Use the right section of your touchscreen to accelerate, and the left to slow down and eventually go backwards.

\section{QA}
\Que{How do you like the concept of the game?}
\Ans{}
\Que{How do you like the steering, and why?}
\Ans{}
\Que{What do you think about the acceleration?}
\Ans{}
\Que{Wat do you think about the connection?}
\Ans{}
\Que{What would you add to this game?}
\Ans{}
\Que{Why would you add that function to the game?}
\Ans{}
\Que{Would this game be more suitable on a tablet or on a phone?}
\Ans{}
\Que{Why do you think so?}
\Ans{}
\Que{What do you think of the presentation of the game?}
\Ans{}
\Que{What do you think of the decision to limit the view of the driver?}
\Ans{}
\Que{Do you think that the player controlling the throttle has as much fun as the steering player?}
\Ans{}
\Que{Why do think that?}
\Ans{}
\Que{We will in the future add mud parts in the road where the car will need to slow down if the want to continue. The player controlling the throtle wil not be able to see this mud, wil this add to the gameplay?}
\Ans{}
\Que{What do you think of the game as it is?}
\Ans{}
\Que{What do you except from the game when it is finished?}
\Ans{}
\Que{Any final suggestions?}
\Ans{}
\end{document}
