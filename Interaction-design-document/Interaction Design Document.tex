\documentclass{article}
\usepackage[utf8]{inputenc}

\title{Interaction Design Document}
\author{Robert Luijendijk}
\date{}

\begin{document}

\maketitle

\section{Introduction}

\section{Musts for a player}
\begin{itemize}
\item It must be clear where the road ends and the grass/fence begins.
\item Lines on the road must be clear and visible
\item The car must be clear and visible.
\item The front of the car must be clear, this is because the players must see where they are going in the beginning.
\item The obstacles must be clearly obstacles.
\item The controls of the car must be clear for each player.
\item The controls for each player must be easy to handle.
\item The start and finish must be clear for the players.
\item In the view for the throttle controlling player the car representation must clear and visible for the player.
\item The car representation in the view for the throttle player must be around the same part of the track as the car from the steering player.
\end{itemize}

\section{Persona's}
\subsection{Clara van Rietgors}
Clara van Rietgors is an happily maried woman who is a housewife with 3 kids. The kids ages raging from 15 to 18 years old, so all of her childeren have smartphones. She has to take the childeren sometimes to the dentist. While at the dentist the childeren wants to play a game on the smartphone. The game is to be played with two persons, the kids want to play with their mother. She wants to play with her childeren, but she does not really know how her phone works. 

\subsection{Buddy de Jong}
Buddy de jong is an 16 year old boy who goes to high school. In the break he wants to play a game, otherwise he is bored. He goes to the cafeteria to eat and hears and sees other people play a game on their smartphone. They make a lot of noise and laugh all the time, because of this Buddy wants to know what that game is. They say it is a game called DuoDrive and playable on the smartphone. Luckily Buddy knows how to handle a phone an downloads the game.

\subsection{Thijs Kurk}
Thijs is a busy student who likes to study, he will make some time to have a good time with his friends or his roommates. Nevertheless he is a busy person, but sometimes he has to cook for his roommates. He is cooking some food that has to boil for some time. Thijs does not like waiting, and wonders what he can do. At that moment his roommate Karl walks in the kitchen, and Thijs remembers a game called DuoDrive. He asks Karl to play that game with him while he waits for the food. Karl downloads the game and plays with Thijs.

\subsection{Lois Cabrera}
Lois is 22 years old and studies Medicine. Tonight she has her second date with Pete Dawson. She knows him from her study. For their second date they are going to watch a movie at her house. She really likes Pete, and hopes that it is going to be a nice date. The movie that they are watching is not so fun as the had hoped. To keep them in the mood, Lois ask if Pete wants to play this fun game she found. Pete downloads it and the have a fun evening.

\subsection{Scott King}
Scott is a corporate business man and he is 35 years old. Every workday of the week he must go to his office on the 60th floor. He goes with the elevator, but it takes a long time. To not get bored, he always plays a game with his colleagues. The game is called DuoDrive, and the game is just short and fun enough to get to the 60th floor.

\section{User stories}
User stories are a great way of making clear how a product works. That's why we have chosen to give a few user stories. 

\subsection{User story 1}
As a player who controls the steering of car, I know I control the controls of the car. I also know I cannot see much of the track, so I cannot see the corners far from advance. My teammate can see them earlier, so if he shouts go to the right I go to the right.

\subsection{User story 2}
As a player who controls the throttle I know I control the speed of the car. While my team is driving, I suddenly see an arrow going to the right. I know I have no control over the steering abilities of the car, so I shout to my teammate that he has to go to the right.

\subsection{User story 3}
As a player who controls the throttle of the car, I know I can control the speed of the car. So when I hear from my teammate that I must go slower, I will go slower.

\subsection{User story 4}
As a player who controls the steering, I know I control the controls of the car. If I see an muddy tile, I know we have to go slower, but I don't control the speed. So I shout to my teammate that he has to go slower.

\subsection{Conclusion}

\end{document}
