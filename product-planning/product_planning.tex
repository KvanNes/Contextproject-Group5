\documentclass{article}
\usepackage[utf8]{inputenc}
\usepackage{glossaries}

\makeglossaries
\newglossaryentry{MoSCoW}{
name = MoSCoW,
description = { Technique to reach a common understanding on the importance of each requirement. M = Must, S = Should, C = Could, W = Wont }
}
\newglossaryentry{acceleration player}{
name = {Acceleration Player},
description = { The player that can only accelerate or decelerate that means, this player cannot steer }
}
\newglossaryentry{steering player}{
name = {Steering Player},
description = { The player that can only steer left or right that means, this player cannot accelerate or decelerate}
}
\newglossaryentry{Gyroscope}{
name = gyroscope,
description = {device used for measuring or maintaining orientation, often built-in to smartphones}
}
\newglossaryentry{Ghosts}{
name = Ghosts,
description = {You see another players shape (ghost) in a transparant color, you cannot collide with it. It is used to show how well your opponents are doing}
}


\title{Product Planning}
\author{Contextgroup 5}
\date{May 2014}

\begin{document}

\maketitle

\section{Introduction}
\section{Product}
\subsection{High-level product backlog}
Our game basically consists of two parts: The \gls{steering player} and the \gls{acceleration player}. The steering player has a limited view of the track and therefore has a hard time steering. In order to accomplish his successful steering this player has to cooperate with the acceleration player which has a complete overview of the track. However this player can only accelerate or decelerate. As both players control the same car they have to be connected in a network. \\\\
Translating the above to a high-level product backlog gives us the following items
\begin{itemize}
	\item Create different interfaces for both players
	\item Create control events for both players
	\item Glue both players with a network.
\end{itemize}
A backlog containing, in detail, all features that can be expected, along with their planning of being implemented is listed below in subsection: Roadmap
\subsection{Roadmap}
This section of the report will contain the features that we designed. We have divided these features according to the \gls{MoSCoW} model to give a brief overview of the features and their priority's. The features that must and should be implemented are planned into sprints, which sprint is stated between brackets.
\subsubsection{Must haves}
The following features are essential and will be implemented.
\begin{itemize}
	\item Race track and Car (Sprint 1)
	\item Network server to implement online player interaction (Sprint 2)
	\item Different controls for different roles (\gls{steering player} and \gls{acceleration player}) including different User Interfaces (Sprint 4)
	\item Road sprites (including tricking ones) (Sprint 1)
    \item Steering player must have minimal vision of the track (Sprint 1)
    \item Acceleration player is able to see the whole track (Sprint 3)
    \item Playable on android (Sprint 5)
\end{itemize}
\subsubsection{Should haves}
The following features are not essential to a minimal game, but will definitely improve gameplay.
\begin{itemize}
    \item Tag teams can match up (Sprint 6)  
    \item Random tracks to delete track learning factor (Sprint 5)
    \item Steering should be done with \gls{Gyroscope} (Sprint 7)
\end{itemize}
\subsubsection{Could haves}
The following features will only be implemented if time allows us to do so.
\begin{itemize}
    \item \gls{Ghosts} of other tag teams to see what position you are in 
    \item Match position during the game
    \item Several car sprites
    \item Track times according to random track length.
    \item Drop objects on track to harass other players.
\end{itemize}
\subsubsection{Wont haves}
These features will not be implemented as they do not improve gameplay relative to the time they cost.
\begin{itemize}
    \item A second mode: ranked mode
    \item Multi-platform interaction.
\end{itemize}
The intended planning will be shown below in a brief overview.
\section{Product Backlog}
For this section, a number of user stories will be told in regards to: features, defects,  technical improvements and know-how acquisition. There will also be a section with the initial release plan.
\subsubsection{User stories of features}
\
\section{Definition of Done}
The last section of this report will explain in detail when the endproduct can be considered done. This section will particularly handle the DoD of a feature, sprint and release.
\subsection{Backlog Items}
Within the backlog an item is considered done, when all the checklist points below are true.
\begin{itemize}
	\item Code complete and approved by lead programmer
	\item Code satisfies coding standards
	\item Unit tests written and executed
	\item Integration system test passed
	\item Documented
\end{itemize}
\subsection{Sprint}
A sprint is considered done, when all the checklist points below are true.
\begin{itemize}
	\item All sprint items considered done
	\item Application is tested globally, all unit tests pass
	\item Integration system test passed
	\item User tests passed
\end{itemize}
\subsection{Releases}
A release is considered done, when all the checklist points below are true.
\begin{itemize}
	\item The product should pass all unit tests
	\item Integration system test passed
	\item Interface looks like the product owner demanded.
	\item (End)user tests passed
	\item Code documented and satisfies coding standards.
\end{itemize}
\clearpage
\printglossaries
\end{document}
