\documentclass[11pt,twoside,a4paper]{article}
\usepackage[english]{babel}
\usepackage{amsmath}
\usepackage{amsthm}
\usepackage{amssymb}

\usepackage[pdfstartview=FitH,pdfpagemode=UseNone]{hyperref}
\usepackage[letterspace=40]{microtype}

\usepackage{a4wide,times}
\usepackage{graphicx}
\usepackage{color}
\usepackage[section]{placeins}

\urlstyle{same}
\linespread{1.1}

\title{Product vision}
\author{
	Pim van den Bogaerdt, pvandenbogaerd, 4215516\\
	Ramin Erfani, 4205502\\
	Robert Luijendijk, 4161467\\
	Mourad el Maouchi, 4204379\\
	Kevin van Nes, kjmvannes, 4020871
}
\begin{document}
\maketitle
\begin{center}
Version 1 (draft)
\end{center}
\clearpage

\section*{Abstract}
In this document we outline answers to five fundamental questions \cite{pichler} about our racing game DuoDrive:

\begin{itemize}
\item Who is going to buy the product? Who is the target customer? 
\item Which customer needs will the product address? 
\item Which product attributes are critical to satisfy the needs selected, and therefore for the success of the product? 
\item How does the product compare against existing products, both from competitors and the same company? What are the product’s unique selling points? 
\item What is the target timeframe and budget to develop and launch the product?
\end{itemize}

For the fourth question we did a small survey of existing racing games available in Google Play \cite{googleplay}.

\clearpage
\tableofcontents

\clearpage

\section{Targets}
The product is focused at people in a waiting room, in a queue, or people waiting to move forward like on a train station. People generally get bored of waiting and thus want some amusement in the meantime. The majority of people who are in such a situation are in the age range of 10+ years. They should have the ability to use a touchscreen device and have proper orientation to make use of the game.

The customers playing the game should be situated in the same room. In this room the equipment needs to be installed so that the users can play together. The places should have space for multiple chairs, in pairs of two, and for two screens where the players can see the gameplay.


\section{Customer needs}
The main customer need the product will address is the need of amusement while bored waiting in a certain room. The customer should have a feeling of pleasure after having played the game. Moreover, since the customer likes time going fast, this product will also address that need. An addictive game will make the player want to play more and more. By gaining this effect, the customer will experience the feeling of time going fast. Without noticing too much, the waiting time is over and the customer is satisfied.

Another important need is achievement. The customer should always have the feeling that he/she has achieved something. Even when not winning, the player will gain the feeling he/she has achieved something. The main achievement is having passed the time in an fun and creative game. Also, with every play the user may get higher on the leaderboard that will be set up. So every run will get the customer more points, allowing him/her to end up higher on the list. The list will also contain the fastest finishing time of to compare with others.


\section{Crucial product attributes}
While waiting, we all want to be able to enjoy the time we are waiting. Therefore, an activity to do is interesting. This can vary in many different forms. Our goal is making this waiting time useful by means of a social activity. DuoDrive will satisfy the customer in that need. Imagine being in a room with multiple people all waiting like you are and all looking for something amusing to do. Having our product in that room will attract them to get up and play.

Because of the effect of sitting back-to-back, an important social aspect needs to be achieved: communicating to win. Interaction is a must so that together the finish line will be reached. Moreover, since you and your partner are not the only ones playing it will become a competition between multiple players. A room filled with eight people, forming four teams of two to get the fastest time, may be possible. Even though you may not know your partner, the customer will try hard, by communicating, to win the game. This will satisfy the need of amusement by social interaction.

Another need of the customer, perhaps the most important one, is amusement in the time in the waiting room. This is a need that must be taken into consideration when developing the product. Because of the social interaction, several things may go very well or rather badly during the gameplay. You might want to say to slow down the car, but accidentally say the opposite, making the car crash. This will create an enormous amount of joy since it was not meant to happen but the crash is funny and both users would actually enjoy it. Also, hearing the other teams scream or notice that they did something wrong will also make your team more confortable. No matter the situation, the chaos that may occur will create a big sense of amusement among you and the others.

Then there is achievement, a feeling that almost everyone should have after having performed an activity. There are several elements processed in the game that will make the player experience this feeling. The first one will be the achievement of communicating in such a way that together you can drive the car over a track. This can be seen as a skill that gets developed through a gaming activity. The second one will be multitasking. Driving a car purely based on another person giving you instructions on when to turn or go straight and on the same moment telling the other how much gas is needed to be given, is difficult. It requires focus and a developed multitasking skill. Being able to do this will give the player a feeling of achievement too. The last, but not least, element will be getting points and get ranked higher on the leaderboard. Everyone who has played a game knows he/she always wants to get a higher score, even though it may take much effort.

Overall, the product contains all the attributes and elements to satisfy the needs of the customer/player. By satisfying these elements it will most certainly guarantee the success of the product. Every player will play the game and leave with a feeling of amusement, and by the time the game has ended, the waiting time will be over. They will have spent their otherwise wasted time playing a game of social interaction, achieving multiple goals and gaining a higher position on the leaderboard.


\section{Comparison with existing products}
In this section we will first discuss two existing racing games and how they compare with ours. We have chosen these games from a mass quantity of games found on Google Play \cite{googleplay}. Our selection is based on visual similarity with what we have in mind for the design of DuoDrive, but with different gameplay. Then we will explain what makes DuoDrive unique.

First, there is Turbo Racer \cite{turboracer}. This is also a top-down racing game; however, it differs from our product in several ways. First of all, this game is single-player whereas ours is multiplayer. The idea of entertaining a group is not apparent; one could play this alone. Also, the cars in Turbo Racer are computer-generated. This means there is no such concept as collaborating or competing. It does have a top-down 2D perspective, though.

Another game is Craigs Race \cite{craigsrace1} \cite{craigsrace2}. This is also a top-down and 2D racing game. In this game you see ghost cars of other players, so it's not completely single-player. However, compared to DuoDrive you do not have to work together in this game either. In fact the game concept is rather plain: you see the road completely and you just have to follow it with the car.

Our product is different in that both collaboration and competition is integrated into the gameplay: in teams of two, you work together to beat the other team(s). Moreover, the collaboration is physical: you are back-to-back and give each other commands. This is what makes DuoDrive much more challenging than existing racing games. We believe we can be trendsetting in developing a game in which close collaboration between two people is a necessity for success.


\section{Timeframe and budget}
{\itshape This section is based on \cite{gamesplanning}.}

For this project a timeframe of ten weeks is available: from April 22 until June 27. We will start creating a prototype in the week of May 5. In particular, on May 23 the first playable must be ready, and the beta and release versions are due on June 6 and June 24, respectively. We will give presentations on May 26 and June 27. Moreover, on June 27 we will demonstrate the product to the public in a game demo market.

Concerning budget, there is no financial budget available. In terms of man-hours, we will be working as a team of five people for roughly 28 hours a week each (calculated as per the ECTS for this course), including lectures.


\clearpage
\begin{thebibliography}{10}

\bibitem{pichler}
Roman Pichler, \textit{Product Vision}. \url{http://www.scrumalliance.org/community/articles/2009/january/the-product-vision}

\bibitem{gamesplanning}
Planning Computer Games project. \url{https://blackboard.tudelft.nl/bbcswebdav/pid-2230195-dt-content-rid-7600571_2/courses/30183-131404/Planning.Games.Project.2014%284%29.pdf?target=blank}

\bibitem{googleplay}
Google Play. \url{https://play.google.com/store/apps}

\bibitem{turboracer}
Turbo Racer (2D car racing). \url{https://play.google.com/store/apps/details?id=com.div.turboracer}

\bibitem{craigsrace1}
Craigs Race. \url{https://play.google.com/store/apps/details?id=com.dlinkddns.craig&hl=nl}

\bibitem{craigsrace2}
Craigs Race Demo. \url{http://www.youtube.com/watch?v=HoFD7q4X21A}

\end{thebibliography}

\end{document}
