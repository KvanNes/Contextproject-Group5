% Based on LaTeX template provided in TI2305 Algoritmiek course

\documentclass[12pt,twoside,a4paper]{article}
\usepackage[english]{babel}
\usepackage{amsmath}
\usepackage{amsthm}
\usepackage{amssymb}

\usepackage[pdfstartview=FitH,pdfpagemode=UseNone]{hyperref}
\usepackage[letterspace=40]{microtype}

\usepackage{a4wide,times}
\usepackage{graphicx}
\usepackage{color}
\usepackage[section]{placeins}

\urlstyle{same}
\linespread{1.1}

\title{Design document}
\author{
	Pim van den Bogaerdt, pvandenbogaerd, 4215516\\
	Ramin Erfani, 4205502\\
	Robert Luijendijk, 4161467\\
	Mourad el Maouchi, 4204379\\
  Kevin van Nes, kjmvannes, 4020871
}
\begin{document}
\maketitle
\begin{center}
Version 1 (draft)
\end{center}
\clearpage

\section*{Abstract}


\clearpage
\tableofcontents

\clearpage

\section{Targets}
The product is focused at people in a waiting room, in a queue, or people waiting to move forward like on a train station. People generally get bored of waiting and thus want some amusement in the meantime. The overhead of people who are in such a situation are in the age range of 10+ years. They should have the ability to use a touchscreen device and have proper orientation to make use of the game.

Even though the product is targeted at such a great audience, the user needs to be at least 18 years old to make the purchase in the Google Play Store. This might also be done by the parents or family of the younger audience. The product aims to let the customer/player gain amusement after having played the game. Since the game has a competitive element in it, the user will experience joy and tries to achieve a higher score or just play again for the taste of revenge. Because of this the customer would like to purchase the product.

\section{Customer needs}
The main customer need the product will address is the need of amusement while bored waiting. The customer should have a feeling of pleasure having played the game. Moreover, since the customer would like the time going fast, this product will also address that need. An addictive game will make the player want to play more and more. By gaining this effect, the customers will experience the feeling of the time going by fast. Without noticing too much, the waiting time would be over and the customer is satisfied.

Another important need is the need of achievement. The customer will always have the feeling that he/she has achieved something. Even when not winning, the player will gain points to unlock several elements in the game. With the coins gained during every game, the player will always achieve something and every gained coin is one more coin towards fun things.


\section{Comparison with existing products}
In this section we will first discuss two existing racing games and how they compare with ours. We have chosen these games from a mass quantity of games found on Google Play \cite{googleplay}. Our selection is based on visual similarity with what we have in mind for the design of DuoDrive, but with different gameplay. Then we will explain what makes DuoDrive unique.

First, there is Turbo Racer \cite{turboracer}. This is also a top-down racing game; however, it differs from our product in several ways. First of all, this game is single-player whereas ours is multiplayer. The idea of entertaining a group is not apparent; one could play this alone. Also, the cars in Turbo Racer are computer-generated. This means there is no such concept as collaborating or competing. It does have a top-down 2D perspective, though.

Another game is Craigs Race \cite{craigsrace1} \cite{craigsrace2}. This is also a top-down and 2D racing game. In this game you see ghost cars of other players, so it's not completely single-player. However, compared to DuoDrive you do not have to work together in this game either. In fact the game concept is rather plain: you see the road completely and you just have to follow it with the car.

Our product is different in that both collaboration and competition is integrated into the gameplay: in teams of two, you work together to beat the other team(s). Moreover, the collaboration is physical: you are back-to-back and give each other commands. This is what makes DuoDrive much more challenging than existing racing games. We believe we can be trendsetting in developing a game in which close collaboration between two people is a necessity for success.


\section{Timeframe and budget}
{\itshape This section is based on \cite{gamesplanning}.}

For this project a timeframe of ten weeks is available: from April 22 until June 27. We will start creating a prototype in the week of May 5. In particular, on May 23 the first playable must be ready and the beta and release versions are due on June 6 and June 24, respectively. We will give presentations on May 26 and June 27. Moreover, on June 27 we will demonstrate the product to the public in a game demo market.

Concerning budget, there is no financial budget available. In terms of man-hours, we will be working as a team of five people for roughly 28 hours a week each (calculated as per the ECTS for this course), including lectures.


\clearpage
\begin{thebibliography}{10}

\bibitem{pichler}
Roman Pichler, \textit{Product Vision}. \url{http://www.scrumalliance.org/community/articles/2009/january/the-product-vision}

\bibitem{gamesplanning}
Planning Computer Games project. \url{https://blackboard.tudelft.nl/bbcswebdav/pid-2230195-dt-content-rid-7600571_2/courses/30183-131404/Planning.Games.Project.2014%284%29.pdf?target=blank}

\bibitem{googleplay}
Google Play. \url{https://play.google.com/store/apps}

\bibitem{turboracer}
Turbo Racer (2D car racing). \url{https://play.google.com/store/apps/details?id=com.div.turboracer}

\bibitem{craigsrace1}
Craigs Race. \url{https://play.google.com/store/apps/details?id=com.dlinkddns.craig&hl=nl}

\bibitem{craigsrace2}
Craigs Race Demo. \url{http://www.youtube.com/watch?v=HoFD7q4X21A}

\end{thebibliography}

\end{document}
