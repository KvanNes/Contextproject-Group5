\documentclass[11pt,twoside,a4paper]{article}
\usepackage[english]{babel}
\usepackage{amsmath}
\usepackage{amsthm}
\usepackage{amssymb}

\usepackage[pdfstartview=FitH,pdfpagemode=UseNone]{hyperref}
\usepackage[letterspace=40]{microtype}

\usepackage{a4wide,times}
\usepackage{graphicx}
\usepackage{color}
\usepackage[section]{placeins}

\urlstyle{same}
\linespread{1.1}

\title{
  Design document\\
  DuoDrive
}
\author{
	Pim van den Bogaerdt, pvandenbogaerd, 4215516\\
	Ramin Erfani, rsafarpourerfa, 4205502\\
	Robert Luijendijk, rluijendijk, 4161467\\
	Mourad el Maouchi, melmaouchi, 4204379\\
	Kevin van Nes, kjmvannes, 4020871
}

\begin{document}

\maketitle
\begin{center}
TI2805 Contextproject, 2013/2014, TU Delft\\
Group 5, Computer Games\\
Version 2
\end{center}
\clearpage

\section*{Abstract}
DuoDrive is a game intended for groups, where players pair up and together try to get their car to the finish. Specifically, one player controls the throttle and the other controls the steer, but they only see information useful for the other player. Thus, close physical collaboration is necessary for success. In this document we outline the context our game is intended for, and answer five fundamental questions (Picher, 2009) about DuoDrive.

First, we will explain the context. Second, we will discuss who is the target customer. Then, we will outline the customer needs the product will address. Subsequently, we state which product attributes are critical to satisfy these needs. Then, we compare our product with two existing products available in Google Play (https://play.google.com/store/apps). This comprises of a limited survey. Lastly, we state the target timeframe and budget available for the product. Each question is discussed in a separate section.

\clearpage
\tableofcontents

\clearpage

\section{Context}
We have designed our game DuoDrive to be played in a waiting room where people get bored. To cure their boredom, these people could play the game. Waiting can cause boredom (Conrad, 1997), and research shows that people gaming may indeed play games when they are bored (Wood, Griffiths, \& Parke, 2007), including young adolescents (Olson et al., 2007). We assume social interaction is something enhancing people's gaming experience, and we have designed DuoDrive with social interaction in mind.

Examples of suitable waiting rooms are at a dentist, in a hospital, or at an airport. Importantly, there must be space for pairs of seats so that collaborating people can sit back-to-back to play the game. We will address the details later.

\section{Targets}
The product is focused at people in a waiting room, in a queue, or people waiting to move forward like on a train station. People generally get bored of waiting so they want some amusement in the meantime. The majority of people who are in such a situation are in the age range of 10+ years. They should have the ability to use a touchscreen device and have proper orientation to make use of the game.

The customers playing the game should be situated in the same room. In this room the equipment needs to be installed so that the users can play together. The places should have space for multiple chairs, in pairs of two, and for two screens where the players can see the gameplay.


\section{Customer needs}
The main customer need the product will address is the need of amusement while bored waiting in a certain room. The customer should have a feeling of pleasure after having played the game. Moreover, since the customer likes time going fast, this product will also address that need. An addictive game will make the player want to play more and more. By gaining this effect, the customer will experience the feeling of time going fast. Without noticing too much, the waiting time is over and the customer is satisfied.

Another important need is achievement. The customer should always have the feeling that he/she has achieved something. Even when not winning, the player will gain the feeling he/she has achieved something. The main achievement is having passed the time by having fun playing a creative game. Also, with every play the user may get higher on the leaderboard that will be set up. So every run will get the customer more points, allowing him/her to end up higher on the list. The list will also contain the fastest finishing times so the customer can compare with others.


\section{Crucial product attributes}
While waiting, we all want to be able to enjoy the time we are waiting. Therefore, an activity to do is interesting. This can vary in many different forms. Our goal is making this waiting time useful by means of a social activity. DuoDrive will satisfy the customer in that need. Imagine being in a room with multiple people all waiting like you are and all looking for something amusing to do. Having our product in that room will attract them to get up and play.

Because of the effect of sitting back-to-back, an important social aspect needs to be achieved: communicating to win. Interaction is a must so that together the finish line will be reached. Moreover, since you and your partner are not the only ones playing it will become a competition between multiple players. A room filled with eight people, forming four teams of two to get the fastest time, may be possible. Even though you may not know your partner, the customer will try hard, by communicating, to win the game. This will satisfy the need of amusement by social interaction.

Another need of the customer, perhaps the most important one, is amusement during the time waiting in the waiting room. This is a need that must be taken into consideration when developing the product. Because of the social interaction, several things may go very well or rather badly during the gameplay. You might want to say to slow down the car, but accidentally say the opposite, making the car crash. This will make for lots of joy since it was not meant to happen but the crash is funny and both users would actually enjoy it. Also, hearing the other teams scream or notice that they did something wrong will also make your team more comfortable. No matter the situation, the chaos that may occur will create a big sense of amusement among you and the others.

Then there is achievement, a feeling that almost everyone should have after having performed an activity. There are several elements processed in the game that will make the player experience this feeling. The first one will be the achievement of communicating in such a way that together you can drive the car over a track. This can be seen as a skill that gets developed through a gaming activity. The second one will be multitasking. Driving a car purely based on another person giving you instructions on when to turn or go straight and on the same moment telling the other how much gas is needed to be given, is difficult. It requires focus and a developed multitasking skill. Being able to do this will give the player a feeling of achievement too. The last, but not least, element will be gaining points and get ranked higher on the leaderboard. Everyone who has played a game knows he/she always wants to get a higher score, even though it may take much effort.

Overall, the product contains all the attributes and elements to satisfy the needs of the customer/player. By satisfying these elements it will most certainly guarantee the success of the product. Every player will play the game and leave with a feeling of amusement, and by the time the game has finished, the waiting time will be over. They will have spent their otherwise wasted time playing a game of social interaction, achieving multiple goals and gaining a higher position on the leaderboard.


\section{Comparison with existing products}
In this section we will first discuss two existing racing games and how they compare with ours. We have chosen these games from a mass quantity of games found on Google Play (https://play.google.com/store/apps). Our selection is based on visual similarity with what we have in mind for the design of DuoDrive, but with different gameplay. Then we will explain what makes DuoDrive unique.

First, there is Turbo Racer (nowkam, 2013). This is, like DuoDrive, a 2D top-down racing game; however, it differs from our product in several ways. First of all, this game is single-player whereas ours is multiplayer. The idea of entertaining a group is not apparent; one could play this alone. Also, the opponents in Turbo Racer are computer-generated. This means there is no such concept as collaborating or competing against another person.

Another game is Craigs Race (Mitchell, 2009). This is also a top-down and 2D racing game. In this game you see ghost cars of other players, so it's not completely single-player. However, compared to DuoDrive you do not have to work together in this game either. In fact the game concept is rather plain: you see the road completely and your car just has to follow it.

Our product is different in that both collaboration and competition are integrated into the gameplay: in teams of two, you work together to beat the other team(s). Moreover, the collaboration is physical: you are back-to-back and give each other commands. This is what makes DuoDrive much more challenging than existing racing games.


\section{Timeframe and budget}
{\itshape This section is based on the official planning document (Blackboard 2014).}

For this project a timeframe of ten weeks is available: from April 22 until June 27. We will start creating a prototype in the week of May 5. In particular, on May 23 the first playable must be ready, and the beta and release versions are due on June 6 and June 24, respectively. We will give presentations on May 26 and June 27. Moreover, on June 27 we will demonstrate the product to the public in a game demo market.

Concerning budget, there is no financial budget available. In terms of man-hours, we will be working as a team of five people for roughly 28 hours a week each (calculated as per the ECTS for this course), including lectures.


\clearpage

\section*{References}

Conrad, P. (1997). It's Boring: Notes on the Meanings of Boredom in Everyday Life. \textit{Qualitative Sociology}. 20(4)465-475.
\newline \newline
Mitchell, C. [Craig Mitchell]. (2009, June 3). \textit{Craigs Race Demo} [Video file]. Retrieved on 11 May, 2014 from \url{http://www.youtube.com/watch?v=HoFD7q4X21A}.
\newline \newline
nowkam. (2013, February 27). \textit{Turbo Racer (2D car racing) - free android game} [Video file]. Retrieved on 11 May, 2014, from \url{http://www.youtube.com/watch?v=iLd3PL7Yxjg}.
\newline \newline
Olson, C. K., Kutner, L. A., Warner, D. E., Almerigi, J. B., Baer, L., Nicholi II, A. M., \& Beresin, E. V. (2007). Factors Correlated with Violent Video Game Use by Adolescent Boys and Girls. \textit{Journal of Adolescent Health}, 41(1)77–83.
\newline \newline
Pichler, R. (2009). The Product Vision. \textit{Scrum Alliance}. Received on 2 May, 2014, from \url{http://www.scrumalliance.org/community/articles/2009/january/the-product-vision}.
\newline \newline
\textit{Blackboard} : Planning Computer Games project (2014). Retrieved on 11 May, 2014 from \url{https://blackboard.tudelft.nl/bbcswebdav/pid-2230195-dt-content-rid-7600571_2/courses/30183-131404/Planning.Games.Project.2014%284%29.pdf?target=blank}.
\newline \newline
Wood, R. T. A., Griffiths, M. D., \& Parke, A. (2007). Experiences of Time Loss among Videogame Players: An Empirical Study. \textit{CyberPsychology \& Behaviour}, 10(1)38-44.

\end{document}
